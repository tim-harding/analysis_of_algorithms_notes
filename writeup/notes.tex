\documentclass[12pt]{article}

\usepackage{fouriernc}
\usepackage{amssymb}
\usepackage{amsmath}
\usepackage{amsfonts}
\usepackage[utf8]{inputenc}
\usepackage[T1]{fontenc}
\usepackage[margin=1in]{geometry}

\newcommand{\curly}[1]{\left\{ #1 \right\}}
\newcommand{\round}[1]{\left(  #1 \right)}
\newcommand{\hard} [1]{\left[  #1 \right]}
\newcommand{\abs}  [1]{\left|  #1 \right|}

\setlength{\parskip}{1em}
\setlength{\parindent}{0in}

\begin{document}

\section{Foundations}

\subsection{The Role of Algorithms in Computing}

\subsection{Getting Started}
\textbf{Loop invariants:} For example, in insertion sort, we work on a subarray of \texttt{0..j} elements. It is a loop invariant that the elements \texttt{0..j-1} are already in sorted order.
\begin{enumerate}
    \item \textbf{Initialization:} It is true prior to the first iteration.
    \item \textbf{Maintenance:} It is true before each iteration.
    \item \textbf{Termination:} It is true when the loop terminates.
\end{enumerate}
In the case of insertion sort, the sorted subarray has only one element, so it must be true on the first iteration. The procedure puts that added element in sorted order within the subarray so that the next iteration will have the loop invariant met. After termination, the loop invariant still holds, which gives the algorithm correctness.

\subsection{Analyzing Algorithms}
The running time $T(n)$ of an algorithm is the cost of each line times the number of times each line is run. This may change depending on the input.

\begin{align*}
    \sum_{j=2}^n j =& \frac{n(n+1)}{2} - 1 \\
    \sum_{j=2}^n (j-1) =& \frac{n(n-1)}{2}
\end{align*}

We mainly focus on worst-case runtime because it gives a good upper bound, because in some cases it occurs fairly often, and because it is often roughly the same as the average case. Average-case runtime also requires more complex probabilistic analysis.

\begin{align*}
    O      &\rightarrow \text{Upper bound} \\
    \Omega &\rightarrow \text{Lower bound} \\
    \Theta &\rightarrow \text{Average case}
\end{align*}

\subsection{Designing Algorithms}

\subsubsection{Divide-and-conquer}
Instead of incremental algorithms such as insertion sort that address a single element at a time, we can also use a divide-and-conquer strategy. Merge sort is a good example.
\begin{enumerate}
    \item \textbf{Divide:} Separate into smaller instances of the same problem.
    \item \textbf{Conquer:} Solve the subproblems recursively. Small enough instances can be solved trivially.
    \item \textbf{Combine:} Combine the subproblem solutions into a solution to the larger problem.
\end{enumerate}

\subsubsection{Analyzing divide-and-conquer}
Given a divide routines $D$ that produces $a$ recurrences of the problem, a conquer routine $C$, and a base case pertaining to $c$ or fewer elements, we have
\begin{align*}
    T(n) &= \begin{cases}
        \Theta(1) & n \leq c \\
        aT\round{\frac{n}{a}} + D(n) + C(n) & \text{else}
    \end{cases}
\end{align*}
The total runtime is the sum of $T(n)$ over each level of the recursion tree.

\end{document}